\subsection{Tuning Rules According to Chien, Hrones and Reswick (Step Response Method)}

This  method is an improvement of the  Ziegler-Nichols  method  and  uses  the
system's step response characteristics $T_u$, $T_g$ and  $K_s$  (discussed  in
section \ref{sec:theory:characterisation}).

The procedure of the Chien-Hrones-Reswick method is:

\begin{enumerate}
    \item Record the step response of the plant and measure the gain $K_{S}$, the delay time $T_{u}$ and the compensation time $T_{g}$.
    \item Determine the controller parameters according to the tuning rules (see tables \ref{tab:CHR_rejection} and \ref{tab:CHR_tracking}).
\end{enumerate}

\begin{center}
    \begin{threeparttable}
        \begin{tabular}{cccc}
            \toprule
            Type & $K_p$                                    &  $T_{i}$            &  $T_{d}$ \\
            \midrule
            P    &  $0.3   \cdot\frac{T_g}{T_u \cdot K_s}$  &  -                  &  -                  \\
            PI   &  $0.6   \cdot\frac{T_g}{T_u \cdot K_s}$  &  $4 \cdot T_u$      &  -                  \\
            PID  &  $0.95  \cdot\frac{T_g}{T_u \cdot K_s}$  &  $2.4 \cdot T_u$    &  $0.42 \cdot T_u$   \\
            \bottomrule
        \end{tabular}
        \caption{Table with controller parameters according to the Chien-Hrones-Reswick method (good disturbance rejection).}
        \label{tab:CHR_rejection}
    \end{threeparttable}
\end{center}

\begin{center}
    \begin{threeparttable}
        \begin{tabular}{cccc}
            \toprule
            Type & $K_p$                                    &  $T_{i}$            &  $T_{d}$ \\
            \midrule
            P    &  $0.3   \cdot\frac{T_g}{T_u \cdot K_s}$  &  -                  &  -                  \\
            PI   &  $0.35  \cdot\frac{T_g}{T_u \cdot K_s}$  &  $1.2 \cdot T_g$    &  -                  \\
            PID  &  $0.6   \cdot\frac{T_g}{T_u \cdot K_s}$  &  $T_g$              &  $0.5 \cdot T_u$    \\
            \bottomrule
        \end{tabular}
        \caption{Table with controller parameters according to the Chien-Hrones-Reswick method (aperiodic behaviour, good tracking).}
        \label{tab:CHR_tracking}
    \end{threeparttable}
\end{center}

Based  on  the  ratio   of   $\lambda=\frac{T_{u}}{T_{g}}$,  we  can  say  how
controllability the control plant is.

\begin{center}
    \begin{threeparttable}
        \begin{tabular}{ccc}
            \toprule
            Ratio                     &  Controllability  & Difficulty \\
            \midrule
            $\lambda < 0.1$           &  very good        & small  \\ 
            $0.1 \leq \lambda < 0.2$  &  good             & medium  \\ 
            $0.2 \leq \lambda < 0.4$  &  fair             & large  \\ 
            $0.4 \leq \lambda < 0.8$  &  bad              & very large  \\ 
            $ \lambda \geq 0.2$       &  very bad         & \makecell{(special \\ measurements \\ required)}  \\ 
            \bottomrule
        \end{tabular}
        \caption{Table showing approximate ratios and how ``good'' they are}
        \label{tab:controllability}
    \end{threeparttable}
\end{center}

