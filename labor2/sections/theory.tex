\section{Theory}
Responsible for the risk of instability of control circuits are the delays. It is therefore always necessary to ensure that a system is stable. To achieve this we need to tune a PID controller.
Tuning a PID controller means to find suitable parameters for the PID controller for a given application so that the application can be very effective. This is achieved by using  Ziegler-Nichols and Chien-Hrones-Reswick.

\subsection{Tuning Rules According to Ziegler-Nichols (Oscillation Method)}
In this method it is assumed that there are higher-orders, which can be described with a series connection of dead time element $T_{t}$ and a $PT_{1}$ element.
\begin{equation}
G_{s}(s) =  \approx \frac{K_{PS}}{1+sT_{1}} \cdot e^{-sT_{1}}
\end{equation}
If the controller is tentative operated with a P-controller, so one can determine the critical $K_{P}$ and the critical period $\tau_{crit}$ by increasing $K_{P}$ until the stability limit is reached.
The procedure of Ziegler-Nichols is:
\begin{enumerate}
\item Use a pure P controller and set a small gain $K_{P}$.
\item Increasing the gain $K_{P}$ until an undamped oscillation occurs.
\item Indentify the critical gain $K_{P,crit}$ and the critical period $\tau_{crit}$
\item According to table \ref{table1} specifying the controllers parameter.
\end{enumerate}

\begin{center}
\begin{tabular}{|c|c|c|c|}
\hline
Controller type & $K_{P}$ & $T_{i}$ & $T_{d}$ \\ \hline
P & $0.5 \cdot K_{P,crit}$ & - & - \\ \hline
PI & $0.45 \cdot K_{P,crit}$ & $0.85 \cdot \tau_{crit}$  & - \\ \hline
PID & $0.6 \cdot K_{P,crit}$ & $0.5 \cdot \tau_{crit}$ & $0.12 \cdot \tau_{crit}$ \\\hline

	\label{table1}
	%\caption{Table with Controller Parameters}
\end{tabular}
\end{center}

The advantage of the Ziegler-Nichols is that he is simple to apply and because of that also easy to understand. On the other side it is very risky because the control loop must be operated close to instability. This can be depending on the system very dangerous and very expensive. Furthermore theoretically not all control plants can be made to oscillate even if in the practice most plants oscillate because of delays and non linearities. The practical significance of this method is thus limited.

\subsection{Tuning Rules According to Chien-Hrones-Reswick (Step Response Method)}
The step response method is an improvement of the Ziegler-Nichols method.
Before using the step response method, first we have to select the step amplitude.
If the step amplitude is too small, the control plant behaves in an atypic way. This happens because of small disturbances which affect the step response. A good example for this is static friction.
On the other side if the step amplitude is too large, then nonlinear effects can affect the behavior of the control plant.
The procedure of Chien-Hrones-Reswick is:
\begin{enumerate}
\item Recording the step response of the control plan and measure the gain $K_{S}$, the delay time $T_{u}$ and the compensation time $T_{g}$.
\item Determine the controller parameters according to the tuning rule (see \ref{table2}
\end{enumerate}

\begin{center}
\begin{tabular}{|c|c|c|c|}
\hline
Controller type & $K_{P}$ & $T_{i}$ & $T_{d}$ \\ \hline
P & $0.3 \cdot\frac{T_{g}}{T_{u}} \cdot \frac{1}{K_{s}}$ & - & - \\ \hline
PI & $0.35 \cdot\frac{T_{g}}{T_{u}} \cdot \frac{1}{K_{s}}$ & $1.2 \cdot T_{g}$  & - \\ \hline
PID & $0.3 \cdot\frac{T_{g}}{T_{u}} \cdot \frac{1}{K_{s}}$ & $T_{g}$ & $0.5 \cdot T_{u}$ \\\hline

	\label{table1}
	%\caption{Table with Controller Parameters(Aperiodic behavior, good tracking)}
\end{tabular}
\end{center}
Based on the ratio of $\frac{T_{u}}{T_{g}}$, we can say how controllability the control plant is.

\begin{center}
\begin{tabular}{|c|c|c|}
\hline
$\frac{T_{u}}{T_{g}}$ & Controllability & Effort \\ \hline
$\frac{T_{u}}{T_{g}} < 0.1$ & very good & small  \\ \hline
$0.1 \leq \frac{T_{u}}{T_{g}} < 0.2$ &  good & medium  \\ \hline
$0.2 \leq \frac{T_{u}}{T_{g}} < 0.4$ &  fair & large  \\ \hline
$0.4 \leq \frac{T_{u}}{T_{g}} < 0.8$ &  bad & very large  \\ \hline
$ \frac{T_{u}}{T_{g}} \geq 0.2$ &  very bad & special measures required  \\ \hline
	\label{table1}
	%\caption{Table with Controller Parameters(Aperiodic behavior, good tracking)}
\end{tabular}
\end{center}